\documentclass[utf8,a4paper]{article}
\usepackage{amsmath}
\usepackage{amsfonts}
\usepackage{amssymb}

\usepackage{mathtools}
\usepackage{amsthm}
\usepackage{stmaryrd}%symbols used so far: \mapsfrom
\usepackage{empheq}

\newtheorem{theorem}{Theorem}[section]
\newtheorem{proposition}{Proposition}[section]
\newtheorem{lemma}{Lemma}[section]
\newtheorem{corollary}{Corollary}[section]

\theoremstyle{definition}
\newtheorem*{definition}{Definition}
\newtheorem*{eg}{Example}
\newtheorem*{remark}{Remark}
\newtheorem*{ex}{Exercise}
\newtheorem*{notation}{Notation}
\newtheorem*{slogan}{Slogan}
\newtheorem*{convention}{Convention}
\newtheorem*{assumption}{Assumption}
\newtheorem*{question}{Question}
\newtheorem*{answer}{Answer}
\newtheorem*{note}{Note}
\newtheorem*{application}{Application}

\newcommand*{\mktitlepage}{\begin{titlepage}
		\begin{center}
			{ \scshape\huge Mathematical Tripos \par }
			\vspace{2cm}
			{\huge Part \npart \par}
			\vspace{0.6cm}
			{\Huge \bfseries \ntitle \par}
			\vspace{1.2cm}
			{\Large\nterm \quad \nyear \par}
			\vspace{2cm}			
			{\large \emph{Lectures by } \par}
			\vspace{0.2cm}
			{\Large \scshape \nlecturer}
			
			\vspace{0.5cm}
			{\large \emph{Notes by }\par}
			\vspace{0.2cm}
			{\Large { \nauthor} }
		\end{center} %
		\title{\ntitle}
		\author{\nauthor}
	\end{titlepage}}


\newcommand*{\conj}[1]{\overline{#1}}
%set complement
\newcommand*{\stcomp}[1]{\overline{#1}}
\newcommand*{\compose}{\circ}
\newcommand*{\nto}{\nrightarrow}
\newcommand*\p{\partial}
%embed
\newcommand*{\embed}{\hookrightarrow}

%matrix
\newcommand*{\matrixring}{\mathcal{M}}

%groups
\newcommand*{\normal}{\trianglelefteq}

%fields
\newcommand*{\C}{\mathbb{C}}
\newcommand*{\R}{\mathbb{R}}
\newcommand*{\Q}{\mathbb{Q}}
\newcommand*{\Z}{\mathbb{Z}}
\newcommand*{\N}{\mathbb{N}}
\newcommand*{\F}{\mathbb{F}}

%asymptotic
\newcommand*{\bigO}{O}
\newcommand*{\smallo}{o}

%probability
\newcommand*{\prob}{\mathbb{P}}
\newcommand*{\E}{\mathbb{E}}

%vector calculus
\newcommand*{\gradient}{\V \nabla}
\newcommand*{\divergence}{\gradient \cdot}
\newcommand*{\curl}{\gradient \cdot}

%quantum
\newcommand*{\curlyH}{\mathcal{H}}
\usepackage{physics}

%opening
\def\ntitle{Cosmology}
\def\nauthor{mostanes}
\def\npart{II}
\def\nterm{Michelmas}
\def\nyear{2018}
\def\nlecturer{}
\begin{document}

\mktitlepage

\newpage

\setcounter{section}{-1}

\section{Preface}
These are some sketches based on the Cosmology lectures. Content that is deemed trivially known is glossed over.

\newpage

\section{List of terms and abbreviations}
\begin{tabular}{ll}
	LSS & Large Scale Structure \\
	AGN & Active Galactic Nuclei \\
	SGR & Soft Gamma Repeater \\
	GRB & Gamma-Ray Burst \\
	CMB & Cosmic Microwave Background \\
	AU  & Astronomical Unit \\
	ly  & light-year \\
\end{tabular}

\section{Introduction}
\subsection{The Expanding Universe}
Properties of the universe: homogeneous and isotropic (on scales $\gg 10$ Mpc).\\
Constituents: Large Scale Structure (Galaxies - AGNs, Quasars, etc.), CMB

\subsection{Units}
\begin{itemize}
	\item time: year, Gy \quad (fun note: $1$ year $\approx \pi 10^7 s$ up to $\sim 0.5\%$ accuracy in the Gregorian calendar)
	\item distances: AU, pc
	\begin{itemize}
		\item pc $\sim$ distance between neighboring stars in a galaxy
		\item Mpc $\sim$ intergalactic distances
		\item Gpc $\sim$ largest structures of the universe
	\end{itemize}
	\item Natural units: $ c = \hbar = k_B = 1 $.
	\item Planck units: Natural units + $ G = 1 $.
\end{itemize}

\subsection{Cosmic distance ladder}
How to measure distances on a cosmic scale? Successive units of comparison:
\begin{itemize}
	\item Cepheid stars
	\item Type IA supernova
	\item Redshift*
\end{itemize}
* - not in the lecture (yet?)

\end{document}