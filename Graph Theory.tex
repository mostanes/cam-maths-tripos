\documentclass[utf8,a4paper]{article}
\usepackage{amsmath}
\usepackage{amsfonts}
\usepackage{amssymb}

\usepackage{mathtools}
\usepackage{amsthm}
\usepackage{stmaryrd}%symbols used so far: \mapsfrom
\usepackage{empheq}

\newtheorem{theorem}{Theorem}[section]
\newtheorem{proposition}{Proposition}[section]
\newtheorem{lemma}{Lemma}[section]
\newtheorem{corollary}{Corollary}[section]

\theoremstyle{definition}
\newtheorem*{definition}{Definition}
\newtheorem*{eg}{Example}
\newtheorem*{remark}{Remark}
\newtheorem*{ex}{Exercise}
\newtheorem*{notation}{Notation}
\newtheorem*{slogan}{Slogan}
\newtheorem*{convention}{Convention}
\newtheorem*{assumption}{Assumption}
\newtheorem*{question}{Question}
\newtheorem*{answer}{Answer}
\newtheorem*{note}{Note}
\newtheorem*{application}{Application}

\newcommand*{\mktitlepage}{\begin{titlepage}
		\begin{center}
			{ \scshape\huge Mathematical Tripos \par }
			\vspace{2cm}
			{\huge Part \npart \par}
			\vspace{0.6cm}
			{\Huge \bfseries \ntitle \par}
			\vspace{1.2cm}
			{\Large\nterm \quad \nyear \par}
			\vspace{2cm}			
			{\large \emph{Lectures by } \par}
			\vspace{0.2cm}
			{\Large \scshape \nlecturer}
			
			\vspace{0.5cm}
			{\large \emph{Notes by }\par}
			\vspace{0.2cm}
			{\Large { \nauthor} }
		\end{center} %
		\title{\ntitle}
		\author{\nauthor}
	\end{titlepage}}


\newcommand*{\conj}[1]{\overline{#1}}
%set complement
\newcommand*{\stcomp}[1]{\overline{#1}}
\newcommand*{\compose}{\circ}
\newcommand*{\nto}{\nrightarrow}
\newcommand*\p{\partial}
%embed
\newcommand*{\embed}{\hookrightarrow}

%matrix
\newcommand*{\matrixring}{\mathcal{M}}

%groups
\newcommand*{\normal}{\trianglelefteq}

%fields
\newcommand*{\C}{\mathbb{C}}
\newcommand*{\R}{\mathbb{R}}
\newcommand*{\Q}{\mathbb{Q}}
\newcommand*{\Z}{\mathbb{Z}}
\newcommand*{\N}{\mathbb{N}}
\newcommand*{\F}{\mathbb{F}}

%asymptotic
\newcommand*{\bigO}{O}
\newcommand*{\smallo}{o}

%probability
\newcommand*{\prob}{\mathbb{P}}
\newcommand*{\E}{\mathbb{E}}

%vector calculus
\newcommand*{\gradient}{\V \nabla}
\newcommand*{\divergence}{\gradient \cdot}
\newcommand*{\curl}{\gradient \cdot}

%quantum
\newcommand*{\curlyH}{\mathcal{H}}
\usepackage{physics}

%opening
\def\ntitle{Graph Theory}
\def\nauthor{mostanes}
\def\npart{II}
\def\nterm{Michelmas}
\def\nyear{2018}
\def\nlecturer{Paul Russel}
\begin{document}
	
	\mktitlepage
	
	\newpage
	
	\setcounter{section}{-1}
	
\section{Preface}
The course is self-contained and has almost no prerequisites (the only ones are mainly IA material). The book recommended for a more in-depth study of the material is I. B. Bollobas's "Modern Graph Theory" (published by Springer).

\newpage

\section{Introduction}
A preliminary definition of graphs:
\begin{definition}
	A graph consists of some 'vertices' with soem pairs of vertices joined by 'edges'
\end{definition}

\subsection{Types of graph problems:}
\subsubsection{Bridges of Konigsberg}
\begin{question}Is it possible to walk round the city crossing each bridge precisely once and returning to the starting point?
\end{question}
\begin{question}
Is it possible to walk round the multigraph traversing each edge precisely once and finishing at the starting vertex?
\end{question}
\subsubsection{Four Colour Problem}
\begin{question}
	How many colours are needed to colour a map?
\end{question}
\subsubsection{Simultaneous coset representation}
Let $G$ be a finite group and $H \leq G$. Let $n = \left|G:H\right|$. Then we know $\exists a_i, b_i \in G$ s.t. $a_i H$ are the left cosets and $H b_i$ are the right cosets of $H$ in $G$.
\begin{question}
	$\exists ? c_i $ s.t. $c_i H$ are the left cosets and $H c_i$ are the right cosets of $H$ in $G$
\end{question}
By considering $X$ the set of left cosets and $Y$ the set of right cosets of $H$ in $G$, let $E = X \cup Y$ and $V = \left\lbrace (gH, Hg) : g \in G \right\rbrace$.
\begin{question}
	$\exists ? \epsilon \subset E$ s.t. $\forall v \in V$, $\exists$ unique $e \in \epsilon$ s.t. $e$ has $v$ as an endpoint
\end{question}
\subsubsection{Fermat equation mod p}
$x^n + y^n = z^n$ has no non-trivial solutions in $\Z$ if $n \geq 3$.
\begin{question}
	Does $x^n + y^n = z^n$ have any non-trivial solutions in $\Z_p$?
\end{question}
\begin{theorem}
	Let $n \in \N$. Then for $\forall$ sufficiently large $p$, there are $x,y,z \neq 0 \pmod{p}$ with $x^n + y^n \equiv z^n \pmod{p}$.
\end{theorem}
\begin{proof}
	Let $G = \Z_p^\times$, the multiplicative group of residues modulo $p$. Let $H = \left\lbrace g^n : g \in G \right\rbrace \leq G$. We want $x, y, z \in H$ s.t. $x+y=z$.\\
	Looking at the cosets of $H$, we have $\left|H\right| \geq \frac{\left|G\right|}{n}$. We also note that $\forall g \in G$, if $\exists u,v,w \in gH$, then $g^{-1}u + g^{-1}v = g^{-1}w$ and $g^{-1}u, g^{-1}, g^{-1}w \in H$. We have reduced the problem to the following combinatorial statement:
\end{proof}
\begin{theorem}[Schur's theorem]
	Let $n$ be a positive integer. Then for $\forall$ sufficiently large $k$, if $\left[k\right] = \left\lbrace 1, 2, ..., k \right\rbrace$ is partitioned in $n$ parts, $\exists u, v, w$ in the same part s.t. $u+v=w$.	
\end{theorem}

\end{document}